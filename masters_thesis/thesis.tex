\documentclass[12pt, twoside]{fithesis2}

% ===== PACKAGES =====
% language settings
\usepackage[english]{babel}
% enabling new fonts support (nicer)
\usepackage{lmodern}
% setting input encoding
\usepackage[utf8]{inputenc}
% setting output encoding
\usepackage[T1]{fontenc}
% fithesis2 requires csquotes
\usepackage{csquotes}
% set page margins
\usepackage[top=3.0cm, bottom=3.5cm, left=2.9cm, right=1.9cm]{geometry}
% package to make bullet list nicer
\usepackage{enumitem}
% math symbols and environments
\usepackage{mathtools}
\usepackage{amsmath}
\usepackage{amssymb}
% packages for complex tablestj
\usepackage{tabularx}
% package for defining new floating environments
\usepackage{float}
\usepackage[labelfont=]{caption}
% package for drawing
\usepackage{tikz}
\usetikzlibrary{shapes,positioning,fit,plotmarks}
% code listings
\usepackage{listings}
% code highlighting
\usepackage[chapter]{minted}

\usepackage{pdfpages}
\usepackage{afterpage}
\usepackage{dirtree}

% space between paragraphs [smaller space between paragraphs]
\setlength{\parskip}{0.6em plus0.2em minus0.2em}

% bibliography management
\usepackage[
  backend=biber, % use biber
  bibstyle=ieee-alphabetic, %IEEE with alphabetic citations
  citestyle=alphabetic, % citation style
  %citestyle=numeric, % citation style
  url=true, % display urls in bibliography
  hyperref=auto, % detect hyperref and create links
]{biblatex}
\addbibresource{thesis.bib}

% break long urls
\setcounter{biburllcpenalty}{7000}
\setcounter{biburlucpenalty}{8000}

% setting custom colors for links
\usepackage{xcolor}
\definecolor{theme-red}{rgb}{0.62,0.01,0.05}
\definecolor{dark-red}{rgb}{0.6,0.15,0.15}
\definecolor{dark-green}{rgb}{0.15,0.4,0.15}
\definecolor{medium-blue}{rgb}{0,0,0.5}
\definecolor{light-gray}{rgb}{0.93,0.93,0.93}

\def\chapterautorefname{Chapter}

% generating hyperlinks in document
\usepackage{url}
\usepackage{xpatch}
\usepackage[
    plainpages=false, % get the page numbering correctly
    pdfpagelabels, % write arabic labels to all pages
    unicode, % allow unicode characters in links
    colorlinks=true, % use colored links instead of boxed
    linkcolor={theme-red},
    citecolor={theme-red},
    urlcolor={theme-red}
]{hyperref}

\providecommand*{\listingautorefname}{listing}

\newcommand\blankpage{%
    \null
    \thispagestyle{empty}%
    \addtocounter{page}{-1}%
    \newpage}

% ===== FI THESIS SETTINGS =====
\thesistitle{Extracting Parts of Programs into Separate Binaries}
\thesissubtitle{Master's thesis}
\thesisstudent{Tomáš Mészaroš}
\thesiswoman{false}
\thesisfaculty{fi}
\thesisyear{2018}
\thesisadvisor{Mgr. Marek Grác, Ph.D.}
\thesislang{en}

% ===== LATEX DOCUMENT SETTINGS =====
% only put chapters and sections into the TOC
\setcounter{tocdepth}{2}

% renew command for shorter and nicer underscore
\renewcommand{\_}{\leavevmode \kern0.07em\vbox{\hrule width0.4em}}

% ===== COMMANDS =====
% define square symbol
\newcommand{\squarebullet}{\textcolor{black}{\raisebox{0.15em}{\rule{4pt}{4pt}}}}
\newcommand{\emptysquarebullet}{\textcolor{black}{\raisebox{0.10em}{\tiny$\square$}}}

\newenvironment{myItemize}{
  \begin{itemize}[
    leftmargin=2em,
    rightmargin=1em,
    itemsep=\parskip,
    parsep=0em,
    topsep=0em,
    partopsep=0em
]
  \renewcommand{\labelitemi}{\squarebullet}
  \renewcommand{\labelitemii}{\textbullet}
}{
  \end{itemize}
}

\newenvironment{myEnumerate}{
  \begin{enumerate}[
    leftmargin=2em,
    rightmargin=1em,
    itemsep=\parskip,
    parsep=0em,
    topsep=0em,
    partopsep=0em
]
}{
  \end{enumerate}
}

% define new environment for code
\lstnewenvironment{code}{
  \lstset{
  frame=lines,
  rulecolor=\color{black},
  basicstyle=\ttfamily,
  columns=fullflexible,
  showspaces=false,
  showstringspaces=false,
  escapeinside={<*}{*>},
  belowskip=0.2em
  }}{}


% ===== BEGIN DOCUMENT =====
\begin{document}

\FrontMatter
\ThesisTitlePage

% zadanie a prehlasenie
% TODO: Comment this for the electronic version!
%\includepdf[pages=-]{blank.pdf}
%\includepdf[pages=-]{zadanie.pdf}
%\includepdf[pages=-]{blank.pdf}
%\includepdf[pages=-]{prehlasenie.pdf}

\begin{ThesisDeclaration}
    \DeclarationText
    \AdvisorName
\end{ThesisDeclaration}

\begin{ThesisThanks}
Mom, Dad, Martin, Marek Grac, Viktor, Pavel, Jozef, Majo, Pato, Mikasa, Astrix,
Misa, Marketa, Tomas, Janka, Hanka, etc.

Dedicated to those who are brave enough to read this stuff.

You are heroes!

\bigskip
\begin{figure}[ht]
    \includegraphics[width=320px]{images/one_does_not_simply.jpg}
\end{figure}

\end{ThesisThanks}

\begin{ThesisAbstract}
    abstract TBA
\end{ThesisAbstract}

\begin{ThesisKeyWords}
    keyword, keyword, keyword, keyword, keyword, keyword
\end{ThesisKeyWords}


\MainMatter
\tableofcontents



% === CHAPTER ==================================================================
\chapter{Introduction: TODO}
\label{chap:intro}

User wants to know the value of some variable in the program. He/she can run
debugger of choice, set breakpoint at the selected variable location and let
debugger execute input program step by step until it reaches the selected
variable. Finally, debugger steps on the targeted variable and thus can extract
its value and provide it back to the user.

The procedure described above is usually part of the standard standard approach
when user want to get value of some selected variable during the program
execution. Unfortunately, this approach is cumbersome in case when user want to
execute above procedure many times. Procedure consists of many manual steps
which is time consuming to perform. Ideally, there could be script that takes
line of code (or variable name) as an input and produces output with the value
of the selected target

Normally, this method would require to use debugger with the scripting support
and write scripts that would instruct debugger what exactly to do, basically
replicating the manual approach.

Instead of scripting debugger to do the extraction, we could write tool that
would accept the same user input as the approach above (line of code/variable
name), run analysis on where the execution flow in the program would occur to
get to the target instruction and transplant subset of the input program into
separate binary.

This way, user will have separate, executable that upon running would produce
value of the targeted instruction, without having to manually step thought or
script debugger.

This thesis aims to devise method and implement this method in a tool for
statically transplanting a subset of a C program. Using the devised method, the
selected program subset should be extracted from the original program provided
by the user and synthesized as an independent, executable binary.

Proposed solution should be implemented in a tool having appropriate form,
either as a standalone application or an LLVM plugin. It should easily accept
user to provide their own input programs.

Finally, tool should be used to test at least two real-world open-source C
programs in order to find where the room for improvements is and what could be
improved in the future.


% summary of the thesis

The following sections of this thesis are structured as follows.
In the \autoref{chap:llvm} we will briefly introduce the LLVM compiler
infrastructure.  Explain what makes it so popular and why we picked this tool
for our implementation.
The following \autoref{chap:design}, we will introduce method that is the basis
of this thesis aim.
We will devote \autoref{chap:implementation} for explaining specific details
and intricacies of implementation.
Experiments and their results will be discussed in the chapter
\autoref{chap:experiments}. Finally \autoref{chap:conclusions} summarizes the
results of this thesis and describes possible further research and development
opportunities.


\chapter{The LLVM Compiler Infrastructure: TODO}
\label{chap:llvm}

"The LLVM (FOOTNOTE:The name "LLVM" itself is not an acronym; it is the full
name of the project.) Project is a collection of modular and reusable compiler
and toolchain technologies." \cite{llvm}

"an umbrella project that hosts and develops a set of close-knit low-level
toolchain components (e.g., assemblers, compilers, debuggers, etc.), which are
designed to be compatible with existing tools typically used on Unix systems"

"the main thing that sets LLVM apart from other compilers is its internal
architecture." [6] Primary subprojects:

LLVM core clang ...  Strengths: "A major strength of LLVM is its versatility,
flexibility, and reusability"

\section{Intermediate Representation}
\label{sec:llvm-ir}

% subsection
Introduction
- IR AKA LLVM assembly language AKA LLVM

- "LLVM is a Static Single Assignment (SSA) based representation that provides
type safety, low-level operations, flexibility, and the capability of
representing ‘all’ high-level languages cleanly. It is the common code
representation used throughout all phases of the LLVM compilation strategy."

- Aims:
 - "The LLVM representation aims to be light-weight and low-level while being
 expressive, typed, and extensible at the same time."

- Representations of IR:
 - as an in-memory compiler IR
 - as an on-disk bitcode representation (suitable for fast loading by a
 Just-In-Time compiler)
 - as a human readable assembly language representation

% subsection
Example of the IR

We have the following C function add

\begin{minted}[ framesep=2mm,
                autogobble,
                frame=lines]{C}
int add(int a, int b) {
    return a+b;
}
\end{minted}

When using clang compiler with -emit-llvm flag, we get the following representation in IR:

\begin{minted}[ framesep=2mm,
                autogobble,
                frame=lines]{llvm}
define i32 @add(i32 %a, i32 %b) #0 {
entry:
  %a.addr = alloca i32, align 4
  %b.addr = alloca i32, align 4
  store i32 %a, i32* %a.addr, align 4
  store i32 %b, i32* %b.addr, align 4
  %0 = load i32, i32* %a.addr, align 4
  %1 = load i32, i32* %b.addr, align 4
  %add = add nsw i32 %0, %1
  ret i32 %add
\end{minted}

% subsection
High Level Structure

- Module structure:
 - functions
 - global variables
 - symbol table entries

- using LLVM linker for module combination
 - we will use this in practice

- Functions:
 - "A function definition contains a list of basic blocks, forming the CFG
 (Control Flow Graph) for the function."
 - PHI nodes


\section{Optimisations}
\label{sec:llvm-opt}

LLVM uses the concept of Passes for the optimisations. Concrete optimisations
are implemented as Passes that work with some portion of program code (e.g.
Module, Function, Loop, etc.) to collect or transform this portion of the code.
\cite{llvm-passes}

There are the following types of passes:

Analysis passes
- Analysis passes collect information from the IR and feed it into the other
passes. They can be also used for the debugging purposes, for example pass that
counts number of functions in the module.

Examples:

- basiccg: Basic CallGraph Construction
- dot-callgraph: Print Call Graph to “dot” file
- instcount: Counts the various types of Instructions

Transform passes
- Transform passes change the program in some way. They can use some analysis
pass that has been ran before and produced some information.

Examples:

- dce: Dead Code Elimination
- loop-deletion: Delete dead loops
- loop-unroll: Unroll loops

Utility passes
- Utility passes do not fit into analysis passes or transform passes categories.

Examples:

- verify: Module Verifier
- view-cfg: View CFG of function
- instnamer: Assign names to anonymous instructions

\section{Clang}
\label{sec:llvm-clang}

"The Clang project provides a language front-end and tooling infrastructure for
languages in the C language family"

Features and Goals (some overview of clang):
- End-User Features
- Utility and Applications
- Internal Design and Implementation

AST
- what is AST
- AST in clang
- Differences between clang AST and other compilers ASTs
- We will not use clangs AST, we will work directly with IR, it better suits
this project

% ==============================================================================
% === CHAPTER: Design of the Method ============================================
\chapter{Extracting Program Subsets}
\label{chap:design}

In this chapter, we introduce the method for extracting parts of programs
from the provided user input.

Starting with the method overview in the \autoref{sec:method_overview} where we
define what is the user input and briefly outline the method itself, what it
does and what are the steps for achieving the final result.

We follow with the example of the method from the user perspective in the
\autoref{sec:method_example}.

After example, we present in detail each major step that is followed.
Starting with computing data dependencies graph
(\autoref{sec:design-dep}).
Following with the procedure for finding connected components in the computed
data dependencies graph (\autoref{sec:design-components}).
Next follows introduction of the call graph (\autoref{sec:design-callgraph}) and
subsequently procedure for finding path from source to target in it
(\autoref{sec:design-path}).
The chapter ends with the section describing methods on eliminating dead
components and functions from the code (\autoref{sec:design-removing}).


% ===== SECTION: Method Overview ===============================================
\section{Method Overview}
\label{sec:method_overview}

% method input definition
Mandatory \textbf{input} for the method is a touple with the following
definition:

$$
input \equiv (code, target)
$$

where:
\begin{myItemize}
\item \textbf{code} is a C program source code compiled into the llvm bytecode.
\item \textbf{target} is an integer value representing line of code from the
C program source code.
\end{myItemize}

We also define \textbf{source} as an entry to the C program
(\mintinline{text}{main} function).

% what the method does

The method determines what parts of the \emph{input} to extract according to
the \emph{source} and \emph{target}.
Procedure subsequently calculates possible execution path up to the
\emph{target} and extracts this execution path into the separate, functioning
executable.

Barring implementation specific details (which are discussed in the
\autoref{chap:implementation}), the method can be summarized by the following
five steps:

\begin{myEnumerate}
\item Compute data dependencies between instructions.
\item Find connected components in the computed data dependencies inside
every function.
\item Construct call graph, mapping between connected components and functions
that are being called from these components.
\item Find path from source to target in the call graph.
\item Eliminate dead components and functions that do not depend on the path.
\end{myEnumerate}

% what is the result of the method
Upon completion of the steps mentioned above, the llvm bytecode is produced as
an \emph{output}.
We can define \textbf{output} as the extracted part of the original program
according to the \emph{source} and \emph{target} while keeping the consistency
of the code intact.
By \textbf{consistency}, we mean that the \emph{output} code is in a such state
that it was possible to be compiled.
\emph{Output} is expected to be runnable the same way as
the original.
\textbf{**TODO? explain more here or in another chapter what it means for IR to
be compiled without problems?**}

% === SECTION: Method Example ==================================================
\section{Example}
\label{sec:method_example}

User provided us with the \emph{input} in the form of the following
C program source code that is stored in the file \mintinline{text}{example.c}:

\begin{minted}[label=example.c,frame=lines,framesep=10pt,linenos]{C}
int foo(int n) {
    int x = n + 10;
    return x;
}

int bar(void) {
    int y = 42;
    return y;
}

int main(void) {
    int some_int = 10;
    int foo_result = foo(some_int);
    int bar_result = bar();
    return 0;
}
\end{minted}

Since our method does not work directly with the C source code but instead
works with the LLVM Intermediate Representation (IR), lets use clang and emit
IR from the presented C source code in order to demonstrate the procedure more
clearly:
\footnote {
Flag \mintinline{text}{-S} tells clang to only run preprocess and
compilation steps, while \mintinline{text}{-emit-llvm} makes sure to use the LLVM
representation for assembler and object files. For detailed description of
various clang flags, visit
\url{https://clang.llvm.org/docs/ClangCommandLineReference.html}.
}

\begin{minted}[frame=lines,framesep=10pt]{bash}
  clang -S -emit-llvm example.c -o example.s
\end{minted}

Emitted LLVM IR is stored in the file \mintinline{text}{example.s} and
has the following structure:\footnote{
Strictly speaking, this is not exactly the IR code that would be emitted by the
clang. We have stripped it out of the module info and comments to make it more
readable. To see the unmodified \mintinline{text}{example.s}, please go to the
\autoref{appendix:example}.
}

\begin{minted}[label=example.s,frame=lines,framesep=10pt,linenos]{llvm}
define i32 @foo(i32 %n) #0 {
entry:
  %n.addr = alloca i32, align 4
  %x = alloca i32, align 4
  store i32 %n, i32* %n.addr, align 4
  %0 = load i32, i32* %n.addr, align 4
  %add = add nsw i32 %0, 10
  store i32 %add, i32* %x, align 4
  %1 = load i32, i32* %x, align 4
  ret i32 %1
}

define i32 @bar() #0 {
entry:
  %y = alloca i32, align 4
  store i32 42, i32* %y, align 4
  %0 = load i32, i32* %y, align 4
  ret i32 %0
}

define i32 @main() #0 {
entry:
  %retval = alloca i32, align 4
  %some_int = alloca i32, align 4
  %foo_result = alloca i32, align 4
  %bar_result = alloca i32, align 4
  store i32 0, i32* %retval, align 4
  store i32 10, i32* %some_int, align 4
  %0 = load i32, i32* %some_int, align 4
  %call = call i32 @foo(i32 %0)
  store i32 %call, i32* %foo_result, align 4
  %call1 = call i32 @bar()
  store i32 %call1, i32* %bar_result, align 4
  ret i32 0
}
\end{minted}

% next point
User also provided the line number \textbf{7} from the
\mintinline{text}{example.c} as the target, which corresponds to the line
\textbf{16} from the \mintinline{text}{example.s}. Source is the main function.

Procedure for finding mapping between C code referenced by the user input and
its analogous IR instruction is implementation detail and is be described
in the \autoref{chap:implementation}.

% next point
When we apply the method on the contents of the
\mintinline{text}{example.s} with respect to the source and target,
we get the result stored in the file
\mintinline{text}{example_extracted.s} with the following code:

\begin{minted}[label=example\_extracted.s,frame=lines,framesep=10pt]{llvm}
define i32 @bar() #0 {
entry:
  %y = alloca i32, align 4
  store i32 42, i32* %y, align 4
  ret i32 %0
}

define i32 @main() #0 {
entry:
  %bar_result = alloca i32, align 4
  %call1 = call i32 @bar()
  store i32 %call1, i32* %bar_result, align 4
  ret i32 0
}
\end{minted}

As we can see from the \mintinline{text}{example.c}, execution path from
the program entry in the \mintinline{text}{main} function (which we will
call \textbf{source}) to the \textbf{target} does not include function
\mintinline{text}{foo} and its associated instructions, they can be removed.
We are left only with function \mintinline{text}{bar} which contains target,
and necessary instructions in the function \mintinline{text}{main} along with
the \mintinline{text}{main} itself.
\\
\\
We can now take \mintinline{text}{example_extracted.s} and recompile it back
into the functioning executable.\footnote{
More about recompilation in the \autoref{chap:implementation}.
}


% === SECTION: Computing Data Dependencies =====================================
\section{Computing Data Dependencies}
\label{sec:design-dep}

% motivation
In order to identify what parts of the IR we can afford to remove, it is
imperative to compute dependencies between instructions.
When removing instructions, we need to preserve consistency of the remaining
code so that it can be later compiled into a functional executable.
To ensure this, we first compute dependencies among instructions.

% definitions
We recognize two types of dependencies between IR instructions: control and data
dependencies.
The following terminology and procedures for computing dependencies that we use
are due to the Marek Chalupa master's thesis \textit{Slicing of LLVM Bitcode}
\cite{dg}.

\begin{myItemize}
\item ``\textbf{Control dependence} explicitly states what nodes are
controlled by which predicate.``
\item ``A \textbf{data dependence} edge is between nodes n and
m iff n defines a variable that m uses and there is no intervening definition
of that variable on some path between n and m. In other words, the definitions
from n reach uses in m.``
\end{myItemize}

% method
The crucial information comes from data dependencies. We need to make sure that
the IR integrity will remain intact after we are done with removing IR
instructions.

The following example demonstrates data dependencies for the
previously presented \mintinline{text}{example.c} source code.
Taking closer look specifically at the function \mintinline{text}{main}:

\begin{minted}[frame=lines,framesep=10pt]{llvm}
define i32 @main() #0 {
entry:
  %retval = alloca i32, align 4
  %some_int = alloca i32, align 4
  %foo_result = alloca i32, align 4
  %bar_result = alloca i32, align 4
  store i32 0, i32* %retval, align 4
  store i32 10, i32* %some_int, align 4
  %0 = load i32, i32* %some_int, align 4
  %call = call i32 @foo(i32 %0)
  store i32 %call, i32* %foo_result, align 4
  %call1 = call i32 @bar()
  store i32 %call1, i32* %bar_result, align 4
  ret i32 0
}
\end{minted}

Taking \mintinline{text}{main} code, we can construct graph G where V is set
of vertices (in our case vertex is instruction) and E is set of edges (in our
case, edge between vertices V1 and V2 represents data dependency between
instruction V1 and V2).
In order to compute data dependencies, we use \mintinline{text}{dg} library.
\cite{dg}
\footnote{
For more info please visit \url{https://github.com/mchalupa/dg} \cite{dg}.
We will take a closer look at \mintinline{text}{dg} in the
\autoref{chap:implementation}.
}

Computed \textbf{data dependency graph} for instructions from the function
\mintinline{text}{main} is presented in the \autoref{fig:data_deps_graph}.
This graph is stored and used in the next step of the method for finding
connected components (\autoref{sec:design-components}).

\begin{figure}[ht]
    \centering
    \includegraphics[]{images/main_dependencies.pdf}
    \caption{Data dependencies graph of the \mintinline{text}{main} function instructions.}
    \label{fig:data_deps_graph}
\end{figure}

Taking a closer look at the instruction:
\mintinline{llvm}{%some_int = alloca i32, align 4}.
We see that the following instructions have data dependency on the
\mintinline{llvm}{%some_int}

\begin{minted}[frame=lines,framesep=10pt, escapeinside=||]{llvm}
  store i32 10, i32* %some_int, align 4
  %0 = load i32, i32* %some_int, align 4
\end{minted}

It is apparent that both instructions need \mintinline{llvm}{%some_int} for
their operand. If we removed
\mintinline{llvm}{%some_int = alloca i32, align 4}
without taking into consideration that there are two
instructions that depend on it, we would get into inconsistent state and two
dependent instructions would contain undefined values as their operand.\footnote{
More about undefined values at
\url{https://llvm.org/docs/LangRef.html\#undefined-values},
\url{https://llvm.org/docs/FAQ.html\#what-is-this-undef-thing-that-shows-up-in-my-code},
\url{https://llvm.org/doxygen/classllvm_1_1UndefValue.html}
}
This would lead into unsuccessful recompilation of the modified code back into
the executable.

% === SECTION: Finding Connected Components ====================================
\section{Finding Connected Components}
\label{sec:design-components}

% motivation
Having shown in the previous section what are inter-instruction data
dependencies and how important they are in relation to the code consistency.
However, they do not fully solve our problem of knowing when it is safe to
remove instruction.
Data dependencies between only two instructions do not reveal the whole picture.
Since we have computed and stored graph of data dependencies, let us propose the
idea of computing connected components of this graph.

% definitions

We define \textbf{connected component} as an isolated subgraph, where
each pair of vertices is connected by some path.

We use \textbf{data dependency graph} computed in the section
\autoref{sec:design-dep} to find its connected components by using the following
algorithm:

\begin{minted}[label=finding components,escapeinside=||,frame=lines,framesep=10pt]{text}
0. Let G = data dependency graph
1. Run |Breadth-first search \cite{clrs}| on G to visit each instruction
2. IF instruction not in any component:
      Create new component and put instruction inside
   ELSE:
      Go to next instruction
\end{minted}

Running the above mentioned algorithm on the \textbf{data dependency graph}
produces connected components for each function in the input.
As an example, we present components for the \mintinline{text}{main} function
in the \autoref{fig:connected_components_graph} (for the clarity, each component
has its own color).

\begin{figure}[ht]
    \centering
    \includegraphics[]{images/main_components.pdf}
    \caption{Connected components of the data dependencies graph
    for the \mintinline{text}{main} function instructions. Individual components
    are differentiated by the color.}
    \label{fig:connected_components_graph}
\end{figure}

% why having components is good

Having instructions within each function separated into connected components
comes useful especially because we can answer the question if some particular
instruction is in data dependency relationship with multiple other instructions
(instructions that form a data dependency path, etc.).

% === SECTION: Computing Call Grap =============================================
\section{Computing Call Graph}
\label{sec:design-callgraph}

%motivation

In respect to the program execution flow, user provided the target and we know
the source.
In order to proceed further, it is needed to compute possible paths from
source to target that could be used by the execution of the program.
Computing and walking the call graph of the program can produce for us this
piece of information.

% definition

A \textbf{call graph}\footnote{
More technically, \emph{call multigraph} \cite{data_flow}.}
is a control flow graph that represents relationship
between program procedures in respect to control flow.\cite{data_flow}
Having call graph $G = \{V, E\}$, set of vertices $V$ typically represents
functions and set of edges $E$ represents transfer of control flow from one
function to another.

In our context, call graph represents relationship between individual
connected components (computed in the \autoref{sec:design-components})
and functions that are being called from these components by one of its
instructions. In other words, our call graph is a set of mappings from
components to the function (or functions).


% example

We construct the call graph using the following algorithm:

\begin{minted}[label=computing call graph,escapeinside=||,frame=lines,framesep=10pt]{text}
0. Let FS = set of functions in the code
1. Let CS(f) = set of components inside function f
2. FOR EACH function F in the set FS:
      FOR EACH component C in the set CS(F):
         FOR EACH instruction I in the component C:
            IF instruction I is a call instruction to some function X:
                Store information X gets called from the C
\end{minted}

Running the presented algorithm on the \textbf{data dependence graph}
of the \mintinline{text}{main} function that we computed earlier
(\autoref{fig:connected_components_graph}) produce call graph structure
shown in the \autoref{fig:call_graph}. We know that in the context of the
\mintinline{text}{main} function, there are four distinct components.
We can see from the computed call graph, that
\mintinline{llvm}{i32 @foo(i32 %n)} is being called from the
\emph{yellow} component, and
\mintinline{llvm}{i32 @bar()} is being called from the \emph{red} component.

\begin{figure}[ht]
    \centering
    \includegraphics[]{images/main_callgraph.pdf}
    \caption{Call graph computed from the connected components
    as shown in the \autoref{fig:connected_components_graph}.}
    \label{fig:call_graph}
\end{figure}


% === SECTION: Finding Path ====================================================
\section{Finding Path from Source to Target}
\label{sec:design-path}

% motivation

Having call graph represented in the structure shown in the
\autoref{fig:call_graph} is beneficial for finding program execution flow path
between specific components within the program in relation to their
dependencies. We can find path from source to target and know which components
this path contains.

% definitions

\emph{''A \textbf{path} is a simple graph whose vertices can be arranged in a linear sequence in
such a way that two vertices are adjacent if they are consecutive in the sequence,
and are nonadjacent otherwise.''}\cite{graph_theory} In our case, linear
sequence of vertices consists of components computed in the
\autoref{sec:design-components}.

% method

The reason why we constructed call graph in the previous chapter is now apparent.
We want to find a path from source to the target and in doing so, know which
connected components are part of this path or not.
Potentially, there may exist infinite number of such paths. From
the optimization standpoint, it would be fitting to find all (or at least as
many as we can) paths and pick some path according to selected optimization
criteria (shortest path, path with smallest connected components, etc.).
However, for our purposes, it will be sufficient to find any path, because
our method does not try to optimize final code in respect to size, speed, etc.

We will use the following algorithm in order to find a path from source to target
in the call graph:

\begin{minted}[label=finding path,escapeinside=||,frame=lines,framesep=10pt]{text}
0. Let G = call graph
1. Run Breadth-first search on G to find path from source to target
\end{minted}

\begin{figure}[ht]
    \centering
    \includegraphics[]{images/main_path.pdf}
    \caption{Path from \textbf{source} to \textbf{target} computed on the
    \autoref{fig:call_graph}. Components in the path are \emph{blue}.
    }
    \label{fig:path}
\end{figure}

Given the source and target, we can see from the \autoref{fig:path} that path
only contains two components, one from \mintinline{text}{main} function and
another from \mintinline{text}{bar}. These two components are going to be the
core of the final, extracted program.

% === SECTION: Eliminating Dead Code ===========================================
\section{Eliminating Dead Components and Functions}
\label{sec:design-removing}

% motivation

This section is the last step that the method performs.
Components and functions that are dead are indentified and removed from the code
and therefore, they will not be part of the final, extracted executable.

% definition
Element of the code marked as \textbf{dead} is defined as such,
that it can be safely deleted without compromising integrity of the code.
In other words, we can safely remove dead elements from the code without being
worried that the execution starting from the source will not reach target.

Having successfully found path from source to target in the last section,
we know that each component that is part of the path cannot be marked as dead.
Therefore, components outside of path have to be explored and the decision
found whether they can be marked as dead and removed or not.

We can use the following basic algorithm for finding and eliminating dead
components:

\begin{minted}[label=eliminating dead components,escapeinside=||,frame=lines,framesep=10pt]{text}
0. Let CS = set of all components in the code
1. Let PATH = set of components on the path from source to target
2. FOR EACH component C in set CS:
      IF component C is not in the set PATH:
         IF component C does not contain terminator:
             Mark component C as DEAD
3. Remove all components marked as DEAD
\end{minted}

% example

After applying the presented algorithm for eliminating dead components on the
code from \mintinline{text}{example.s} we get the result that is visualised
in the \autoref{fig:removing_prepare}.
We can see components marked as dead are colored \emph{red} because they are not
part of the path.
The \emph{yellow} component with the single instruction stands out.
This component is not marked as dead and will not be removed because it contains
terminator instruction \mintinline{llvm}{ret i32 0}.\footnote{
\url{https://llvm.org/doxygen/classllvm_1_1TerminatorInst.html}
More about terminators and why they need special treatment in the
\autoref{chap:implementation}.
}
Finally, all components marked as dead are removed and we are left with the
code that we presented in the \autoref{sec:method_example}
(\mintinline{text}{example_extracted.s}).
Contents of the \mintinline{text}{example_extracted.s} are visualized in the
\autoref{fig:removing_done}.

\begin{figure}[ht]
    \centering
    \includegraphics[]{images/main_removing_prepare.pdf}
    \caption{
    \mintinline{text}{example.s}:
    Components selected for removal are marked \emph{red}.
    \emph{Yellow} component is not marked for removal, because it contains
    terminator.
    }
    \label{fig:removing_prepare}
\end{figure}

\begin{figure}[ht]
    \centering
    \includegraphics[]{images/main_removing_done.pdf}
    \caption{
    \mintinline{text}{example_extracted.s}:
    Final state after removing dead components and functions.
    \mintinline{text}{example.s}.
    }
    \label{fig:removing_done}
\end{figure}

% conclusion

Unfortunately, the basic algorithm presented above works correctly only with the
non-complex inputs that are similar to the \mintinline{text}{example.s}.
We present two extensions of the \mintinline{text}{example.s} program in the
\autoref{subsec:path_function} and \autoref{subsec:path_branching}
along with the improvements of the basic algorithm that cover these two
extensions.


% === SUB-SECTION: External Function ===========================================
\subsection{Path Depending on the External Function}
\label{subsec:path_function}

We modify the \mintinline{text}{example.c} by introducing new function
\mintinline{C}{int qux(void)}. Also, we change line 7 of the
\mintinline{text}{example.c} from
\mintinline{C}{int y = 42} to \mintinline{C}{int y = qux()} and save these
modifications to the \mintinline{text}{example_mod1.c}:

% mod1 C code
\begin{minted}[label=example\_mod1.c,frame=lines,framesep=10pt,linenos]{C}
int foo(int n) {
    int x = n + 10;
    return x;
}

int qux(void) {
    return 42;
}

int bar(void) {
    int y = qux();
    return y;
}

int main(void) {
    int some_int = 10;
    int foo_result = foo(some_int);
    int bar_result = bar();

    return 0;
}
\end{minted}

Target is the same (\mintinline{C}{int y = ...})
as in the original example from the
\autoref{sec:method_example} (line 7 in the \mintinline{text}{example.c}).
In the \mintinline{text}{example_mod1.c}, target sits at the line 11.

\mintinline{text}{example_mod1.c} compiled into the LLVM IR
is stored in the \mintinline{text}{example_mod1.s}.
Accordingly, the call graph with computed components and path for the
\mintinline{text}{example_mod1.s} can be seen in the
\autoref{fig:mod1_prepare}.

When looking upon \autoref{fig:mod1_prepare}, we see that function bar calls
function qux, but qux is not part of the computed path (only components with
the blue color are part of the path, that is one component from main and one
from foo). This is problematic for the basic algorithm that we introduced
in the \autoref{sec:design-removing}. This basic algorithm would mark
qux as dead and subsequently remove this function. However, this would break
code integrity, because function qux has to be called in order for the execution
to correctly proceed.
Therefore, we propose modified algorithm for eliminating dead components that
will take into the account the above described possibility.

\begin{minted}[label=eliminating dead components - mod1,escapeinside=||,frame=lines,framesep=10pt]{text}
0. Let CS = set of all components in the code
1. Let PATH = set of components on the path from source to target
2. Recursively find all called functions that originate from PATH
   using Breath-first search and add them to the PATH.
3. FOR EACH component C in set CS:
      IF component C is not in the set PATH:
         IF component C does not contain terminator:
             Mark component C as DEAD
4. Remove all components marked as DEAD
\end{minted}

By using the above described algorithm instead of the basic one from the
\autoref{sec:design-removing}, the path will contain function qux and therefore,
function qux will not be marked as dead and removed.
We can see the result of the algorithm in the \autoref{fig:mod1_done} with the
corresponding code in the \mintinline{text}{example_mod1_extracted.s}.
Function qux has not been removed which mean that the integrity of the code
was maintained and executable could be successfully produced.

% mod1 IR
\begin{minted}[label=example\_mod1.s,frame=lines,framesep=10pt,linenos]{llvm}
define i32 @foo(i32 %n) #0 {
entry:
  %n.addr = alloca i32, align 4
  %x = alloca i32, align 4
  store i32 %n, i32* %n.addr, align 4
  %0 = load i32, i32* %n.addr, align 4
  %add = add nsw i32 %0, 10
  store i32 %add, i32* %x, align 4
  %1 = load i32, i32* %x, align 4
  ret i32 %1
}

define i32 @qux() #0 {
entry:
  ret i32 42
}

define i32 @bar() #0 {
entry:
  %y = alloca i32, align 4
  %call = call i32 @qux()
  store i32 %call, i32* %y, align 4
  %0 = load i32, i32* %y, align 4
  ret i32 %0
}

define i32 @main() #0 {
entry:
  %retval = alloca i32, align 4
  %some_int = alloca i32, align 4
  %foo_result = alloca i32, align 4
  %bar_result = alloca i32, align 4
  store i32 0, i32* %retval, align 4
  store i32 10, i32* %some_int, align 4
  %0 = load i32, i32* %some_int, align 4
  %call = call i32 @foo(i32 %0)
  store i32 %call, i32* %foo_result, align 4
  %call1 = call i32 @bar()
  store i32 %call1, i32* %bar_result, align 4
  ret i32 0
}
\end{minted}

% mod1 callgraph
\begin{figure}[ht]
    \centering
    \includegraphics[]{images/example_mod1/example_mod1_removing_prepare.pdf}
    \caption{
    \mintinline{text}{example_mod1.s}:
    Components selected for removal are marked \emph{red}.
    \emph{Yellow} component is not marked for removal, because it contains
    terminator. \emph{Green} component was discovered by the modified algorithm
    and has been added to the path and therefore will not be removed.
    }
    \label{fig:mod1_prepare}
\end{figure}

% mod1 extracted
\begin{minted}[label=example\_mod1\_extracted.s,frame=lines,framesep=10pt,linenos]{llvm}
define i32 @qux() #0 {
entry:
  ret i32 42
}

define i32 @bar() #0 {
entry:
  %y = alloca i32, align 4
  %call = call i32 @qux()
  store i32 %call, i32* %y, align 4
  %0 = load i32, i32* %y, align 4
  ret i32 %0
}

define i32 @main() #0 {
entry:
  %bar_result = alloca i32, align 4
  %call1 = call i32 @bar()
  store i32 %call1, i32* %bar_result, align 4
  ret i32 0
}
\end{minted}

% mod1 extracted graph
\begin{figure}[ht]
    \centering
    \includegraphics[]{images/example_mod1/example_mod1_removing_done.pdf}
    \caption{
    \mintinline{text}{example_mod1_extracted.s}:
    Final state after removing dead components and functions from
    \mintinline{text}{example_mod1.s}.
    }
    \label{fig:mod1_done}
\end{figure}


% === SUB-SECTION: Branching Instruction =======================================
\subsection{Path Depending on the Branching Instruction}
\label{subsec:path_branching}

To increase input complexity, lets take \mintinline{text}{example_mod1.c} and
add branching inside main function as shown in the
\mintinline{text}{example_mod2.c}.
After taking look at the generated LLVM IR shown in the
\mintinline{text}{example_mod2.s}, we can see branch instruction
at the line 42.
Also, there are present two new basic blocks that this branch instruction
refers to:
\mintinline{text}{if.then} and \mintinline{text}{if.end} (lines 44 and 49).

If we take a closer look at the generated call graph for
\mintinline{text}{example_mod2.s}
(\autoref{fig:mod2_prepare}), we can observe that the branching instruction
\mintinline{llvm}{br i1 %cmp, label %if.then, label %if.end} is in the
component that is not part of the path. This is problematic, because if we
look at the
\mintinline{text}{example_mod2.c}.
we can clearly see, that in order to get to the target (line 11),
branching needs to be executed and thus included in the path.
We present the algorithm from the previous subsection
(\autoref{subsec:path_function}) with the extension that handles the above
described scenario.

\begin{minted}[label=eliminating dead components - mod2,breaklines,escapeinside=||,frame=lines,framesep=10pt]{text}
0. Let CS = set of all components in the code
1. Let PATH = set of components on the path from source to target
2. FOR EACH component C in PATH:
      FOR EACH instruction I in C:
          IF instruction I is part of basic block handled by the branch instruction:
             FIND component with the branching instruction responsible for I and add it to the PATH
3. Recursively find all called functions that originate from PATH
   using Breath-first search and add them to the PATH.
4. FOR EACH component C in set CS:
      IF component C is not in the set PATH:
         IF component C does not contain terminator:
             Mark component C as DEAD
5. Remove all components marked as DEAD
\end{minted}

As we can see in the
\mintinline{text}{example_mod2.s}, if we examine instruction from the line 45,
we can see that it belongs to the basic block
\mintinline{text}{if.then}. This \mintinline{text}{if.then} basic block is
handled by the branch instruction from the line 42. Therefore, algorithm will
find component that contains this branch instruction and adds it to the path.
The call graph with computed components and path that the algorithm takes as
an input is shown in the \autoref{fig:mod2_prepare}.
Component in the main function marked as green contains branching instruction
that the algorithm identified as needed and therefore added to the path.

The result of the algorithm are presented in the
\mintinline{text}{example_mod2_extracted.s} and visually in the
\autoref{fig:mod2_done}.


% mod2 C code

\begin{minted}[label=example\_mod2.c,frame=lines,framesep=10pt,linenos]{C}
int foo(int n) {
    int x = n + 10;
    return x;
}

int qux(void) {
    return 42;
}

int bar(void) {
    int y = qux();
    return y;
}

int main(void) {
    int some_int = 10;
    int foo_result = foo(some_int);
    int n = 10;
    if (n < 42) {
        int bar_result = bar();
    }
    return 0;
}
\end{minted}

% mod2 IR

\begin{minted}[label=example\_mod2.s,frame=lines,framesep=10pt,linenos]{llvm}
define i32 @foo(i32 %n) #0 {
entry:
  %n.addr = alloca i32, align 4
  %x = alloca i32, align 4
  store i32 %n, i32* %n.addr, align 4
  %0 = load i32, i32* %n.addr, align 4
  %add = add nsw i32 %0, 10
  store i32 %add, i32* %x, align 4
  %1 = load i32, i32* %x, align 4
  ret i32 %1
}

define i32 @qux() #0 {
entry:
  ret i32 42
}

define i32 @bar() #0 {
entry:
  %y = alloca i32, align 4
  %call = call i32 @qux()
  store i32 %call, i32* %y, align 4
  %0 = load i32, i32* %y, align 4
  ret i32 %0
}

define i32 @main() #0 {
entry:
  %retval = alloca i32, align 4
  %some_int = alloca i32, align 4
  %foo_result = alloca i32, align 4
  %n = alloca i32, align 4
  %bar_result = alloca i32, align 4
  store i32 0, i32* %retval, align 4
  store i32 10, i32* %some_int, align 4
  %0 = load i32, i32* %some_int, align 4
  %call = call i32 @foo(i32 %0)
  store i32 %call, i32* %foo_result, align 4
  store i32 10, i32* %n, align 4
  %1 = load i32, i32* %n, align 4
  %cmp = icmp slt i32 %1, 42
  br i1 %cmp, label %if.then, label %if.end

if.then:                                          ; preds = %entry
  %call1 = call i32 @bar()
  store i32 %call1, i32* %bar_result, align 4
  br label %if.end

if.end:                                           ; preds = %if.then, %entry
  ret i32 0
}
\end{minted}

% mod2 callgraph

\begin{figure}[ht]
    \centering
    \includegraphics[]{images/example_mod2/example_mod2_removing_prepare.pdf}
    \caption{
    \mintinline{text}{example_mod2.s}:
    Components selected for removal are marked \emph{red}.
    \emph{Yellow} component is not marked for removal, because it contains
    terminator. \emph{Green} components were discovered by the modified
    algorithm and were added to the path and therefore will not be removed.
    }
    \label{fig:mod2_prepare}
\end{figure}

% mod2 extracted

\begin{minted}[label=example\_mod2\_extracted.s,frame=lines,framesep=10pt,linenos]{llvm}
define i32 @qux() #0 {
entry:
  ret i32 42
}

define i32 @bar() #0 {
entry:
  %y = alloca i32, align 4
  %call = call i32 @qux()
  store i32 %call, i32* %y, align 4
  %0 = load i32, i32* %y, align 4
  ret i32 %0
}

define i32 @main() #0 {
entry:
  %n = alloca i32, align 4
  %bar_result = alloca i32, align 4
  store i32 10, i32* %n, align 4
  %0 = load i32, i32* %n, align 4
  %cmp = icmp slt i32 %0, 42
  br i1 %cmp, label %if.then, label %if.end

if.then:                                          ; preds = %entry
  %call1 = call i32 @bar()
  store i32 %call1, i32* %bar_result, align 4
  br label %if.end

if.end:                                           ; preds = %if.then, %entry
  ret i32 0
}
\end{minted}

% mod2 extracted graph

\begin{figure}[ht]
    \centering
    \includegraphics[]{images/example_mod2/example_mod2_removing_done.pdf}
    \caption{
    \mintinline{text}{example_mod2_extracted.s}:
    Final state after removing dead components and functions from
    \mintinline{text}{example_mod2.s}.
    }
    \label{fig:mod2_done}
\end{figure}

% === CHAPTER: Implementation ==================================================
\chapter{Implementation: TODO}
\label{chap:implementation}

%section
\section{APEX}

\subsection{what is APEX}

\subsection{two components of apex}
\begin{myEnumerate}
\item launcher
\item APEXPass
\end{myEnumerate}

% section
\section{Launcher}

\subsection{compiling input to bytecode with dbg symbols}

\subsection{running basic opts}

\subsection{linking input with apexlib}

\subsection{running apexpass}

\subsection{exporting call graph, dependency graph, disassembly bc}

\subsection{running final binaries}

% section
\section{APEXPass}

\subsection{User Inpur Parsing, Mapping C code to IR}

\subsection{Compute data dependencies between instructions.}
- running dg, init apexdg

\subsection{Find connected components in the apexdg.}

\subsection{Construct call graph, mapping between connected components and functions
that are being called from these components.}

\subsection{Find path from source to target in the call graph.}

\subsection{Eliminate dead components and functions that do not depend on the path.}
The simplest and seemingly correct way would be to remove every connected
component that is not part of the path that we calculated in the earlier
chapter.

This approach would unfortunately produce inconsistent IR. It not enough to
remove only components in the path. We need to include every other component
that is dependent on any other component that is already part of the path.

Checking if we have any branching dependent on the @path
\begin{myEnumerate}
\item investigating block, collecting basic blocks
\item block has no instruction in "if.*" basic block
\item block has some instruction in "if.*" basic block
\item Find branch instruction that services this BB and add block associated with this branch instruction to the @path.
\end{myEnumerate}

Computing what dependency blocks we want to keep
\begin{myEnumerate}
\item marking every block from @path as to keep
\item Mark as visited to make sure we do not process this block in BFS.
\item setting up initial queue for BFS search
\item Go over @path and figure out if there are any calls outside the @path. If there
are, put those called blocks for investigation into the @queue.
\item running BFS
\item Run BFS from queue and add everything for keeping that is not visited.
\item Collecting everything that we do not want to keep
\item We store blocks and functions that we want to remove into sets.
\end{myEnumerate}

Removing unwanted blocks
\begin{myEnumerate}
\item Remove instructions that we stored earier. Watch out for terminators (do not
remove them).
\item Do not erase instruction that is inside target instructions (We need those
instructions intact.)
\end{myEnumerate}

\subsection{Injecting extraction and exit}

\subsection{Recompilation, stripping debug symbols}

%section
\section{Recompilation back to the executable}

% === CHAPTER: Experiments =====================================================
\chapter{Experiments: TODO}
\label{chap:experiments}


% === CHAPTER: Conclusions =====================================================
\chapter{Conclusions: TODO}
\label{chap:conclusions}

% === SECTION ===
\section{Summary of the Results}
\label{sec:conclusions-summary}

% focus on our specific contributions
% focus on the wider view, what this thesis brought to the world

% === SECTION ===
\section{Further Research and Development}
\label{sec:conclusions-next}

% what could be improved theoretically? (e.g. multiple paths, etc.)
% what could be improved technically? (e.g. better json dump)


% === BIBLIOGRAPHY =============================================================
\appendix

% print complete bibliography
\printbibliography

% === CHAPTER ==================================================================
\chapter{Archive structure}
\label{appendix:archive}

Content of the attached archive:

TBA TBA TBA

% === REMOVE THIS BEFORE PRINTING ==============================================
\chapter{Outline}

Extracting Parts of Programs into Separate Binaries

\begin{myEnumerate}
\item Get acquainted with means of the compilation of C programs using the LLVM
compiler infrastructure - clang, LLVM Internal Representation, AST, LLVM
optimizations.
\item Propose a solution to statically transplant a subset of a C program. This
subset should be extracted from the original program and synthesized as an
independent binary.
\item Design and implement the proposed solution in a tool having an
appropriate form (a standalone application or an LLVM plugin).
\item Test the implemented tool on at least 2 real-world open-source C
programs.
\end{myEnumerate}

\noindent\rule{\textwidth}{1pt}

\begin{myItemize}

\item Introdution
    \begin{myItemize}
    \item Give introduction to wider context
    \item Clearly explain aim of the thesis
    \item Give outline of the following chapters
    \end{myItemize}

\item The LLVM Compiler Infrastructure
    \begin{myItemize}
    \item IR
    \item Optimizations
    \item clang
    \end{myItemize}

\item Extracting Program Subsets
    \begin{myItemize}
    \item Intro
    \item
    \item Computing Data Dependencies
    \item Finding Connected Components
    \item Constructing Call Graph
    \item Finding Path
    \item Removing Unnecessary Parts
    \end{myItemize}

\item Implementation
    \begin{myItemize}
    \item APEX
    \item APEXPass
    \item Input Source Code
    \item Parsing User Input (Locating Target Instructions)
    \item Computing Dependencies using dg
    \item Extracting Target Data (Injecting Exit and Extraction)
        \begin{myItemize}
        \item Stripping debug symbols
        \end{myItemize}
    \end{myItemize}

\item Experiments
    \begin{myItemize}
    \item Experiment 1
    \item Experiment 2
    \item Experiment 3
    \end{myItemize}

\item Conclusion
    \begin{myItemize}
    \item Show our contribution to the problem
    \item Show wider image in context to this thesis
    \end{myItemize}

\end{myItemize}



% === CHAPTER ==================================================================
\chapter{\mintinline{text}{example.s}}
\label{appendix:example}

\begin{minted}[frame=lines, framesep=10pt, breaklines,linenos]{llvm}
; ModuleID = 'example.c'
source_filename = "example.c"
target datalayout = "e-m:e-i64:64-f80:128-n8:16:32:64-S128"
target triple = "x86_64-unknown-linux-gnu"

; Function Attrs: noinline nounwind optnone uwtable
define i32 @foo(i32 %n) #0 {
entry:
  %n.addr = alloca i32, align 4
  %x = alloca i32, align 4
  store i32 %n, i32* %n.addr, align 4
  %0 = load i32, i32* %n.addr, align 4
  %add = add nsw i32 %0, 10
  store i32 %add, i32* %x, align 4
  %1 = load i32, i32* %x, align 4
  ret i32 %1
}

; Function Attrs: noinline nounwind optnone uwtable
define i32 @bar() #0 {
entry:
  %y = alloca i32, align 4
  store i32 42, i32* %y, align 4
  %0 = load i32, i32* %y, align 4
  ret i32 %0
}

; Function Attrs: noinline nounwind optnone uwtable
define i32 @main() #0 {
entry:
  %retval = alloca i32, align 4
  %some_int = alloca i32, align 4
  %foo_result = alloca i32, align 4
  %bar_result = alloca i32, align 4
  store i32 0, i32* %retval, align 4
  store i32 10, i32* %some_int, align 4
  %0 = load i32, i32* %some_int, align 4
  %call = call i32 @foo(i32 %0)
  store i32 %call, i32* %foo_result, align 4
  %call1 = call i32 @bar()
  store i32 %call1, i32* %bar_result, align 4
  ret i32 0
}

attributes #0 = { noinline nounwind optnone uwtable "correctly-rounded-divide-sqrt-fp-math"="false" "disable-tail-calls"="false" "less-precise-fpmad"="false" "no-frame-pointer-elim"="true" "no-frame-pointer-elim-non-leaf" "no-infs-fp-math"="false" "no-jump-tables"="false" "no-nans-fp-math"="false" "no-signed-zeros-fp-math"="false" "no-trapping-math"="false" "stack-protector-buffer-size"="8" "target-cpu"="x86-64" "target-features"="+fxsr,+mmx,+sse,+sse2,+x87" "unsafe-fp-math"="false" "use-soft-float"="false" }

!llvm.module.flags = !{!0}
!llvm.ident = !{!1}

!0 = !{i32 1, !"wchar_size", i32 4}
!1 = !{!"clang version 5.0.1 (tags/RELEASE_500/final)"}
\end{minted}





% === END ======================================================================
\end{document}
